\documentclass{article}

\usepackage{minted}
\usepackage{listings}
\usepackage{hyperref}
\hypersetup{
    colorlinks=true,
    linkcolor=blue,
    filecolor=magenta,      
    urlcolor=cyan
}

\begin{document}

\title{Creating a Stand-Alone Spark Environment in Windows Subsystem for Linux}
\author{Abraham Vargas}
\maketitle

\newpage
\tableofcontents
\newpage

\section{\emph{Windows Subsystem for Linux}}
\emph{Windows Subsystem for Linux} (\emph{WSL}) allows \emph{Linux} a distribution (distro) to
run concurrently with \emph{Windows 10}. \emph{WSL 2} now includes a \emph{Linux}
kernel and is several times faster than the original \emph{WSL}. This guide assumes that \emph{WSL 2}
is installed and is running the \emph{Ubuntu 18.04} distro. To install \emph{WSL 2}, follow the
\href{https://docs.microsoft.com/en-us/windows/wsl/install-win10}{Windows Subsystem for Linux Installation Guide for Windows 10}.
Make sure to use \emph{Ubuntu 18.04} as the distro.

\section{Java}
\emph{Spark}, \emph{Hadoop}, and \emph{Hive} all require a \emph{Java Runtime Environment}(\emph{JRE}).
\emph{OpenJDK} is a standard package in most \emph{Linux} distributions. Though \emph{OpenJDK 11}
is the latest version, \emph{Spark} and its related components require \emph{OpenJDK 8}.

    \subsubsection{Install \emph{JRE}}
    To install \emph{OpenJDK 8}, run the following command in a terminal:
    \begin{minted}{bash}
        sudo apt install openjdk-8-jre-headless            
    \end{minted}

    \subsubsection{Set \emph{JAVA\_HOME}}
    \emph{Spark} and other components will need to know the path of the
    \emph{JRE}. This is accomplished by setting the \emph{JAVA\_HOME}
    system variable.

    \begin{description}

        \item[Open] the \emph{.bashrc} file with any editor (e.g., \emph{vim}, \emph{nano}).
        For example:
        \begin{minted}{bash}
            vim ~/.bashrc
        \end{minted}

        \item[Add] the following line to the \emph{.bashrc} file:
        \begin{minted}{bash}
            export JAVA_HOME=/usr/lib/jvm/java-8-openjdk-amd64                 
        \end{minted}
        
        \item[Reload] shell environment configuration:
        \begin{minted}{bash}
            source ~/.bashrc
        \end{minted}

        \item[Verify] that path was correctly set:
        \begin{minted}{bash}
            echo $JAVA_HOME
        \end{minted}
    \end{description}

\section{\emph{PostgreSQL}}
\emph{Hive} requires a database for its metastore. In terms of open-source options, \emph{Hive} can use
\emph{MySQL}, \emph{PostgreSQL}, and \emph{Derby}. This guide will use \emph{PostgreSQL}.

    \subsection{Install \emph{PostgreSQL}}
    \emph{PostgreSQL 10} is the default version packaged for \emph{Ubuntu 18.04}. Install
    \emph{PostgreSQL 10} with the command:
    \begin{minted}{bash}
        sudo apt install postgresql-10
    \end{minted}

    \subsection{Start database server}
    The \emph{PostgreSQL} will need to be started every time \emph{WSL} is started. Initialize the
    server using the command:
    \begin{minted}{bash}
        sudo service postgresql start
    \end{minted}

    \subsection{Configure \emph{Hive} metastore}
    The \emph{psql} application is used to interact with \emph{PostgreSQL} in a terminal.
    \begin{description}
        \item[Log into] \emph{PostgreSQL} via \emph{psql} with default user \emph{postgres}:
        \begin{minted}{bash}
            sudo -u postgres psql 
        \end{minted}

        \item[Add] new user \emph{hive}
        (for simplicity, the new user will also be given \emph{hive} as the password):
        \begin{minted}{postgres}
            CREATE USER hive WITH PASSWORD 'hive';
        \end{minted}

        \item[Create] new database for \emph{Hive} metastore:
        \begin{minted}{postgres}
            CREATE DATABASE hive_metastore;
        \end{minted}

        \item[Give] ownership of \emph{hive\_metastore} database to user \emph{hive}:
        \begin{minted}{postgres}
            ALTER DATABASE hive_metastore OWNER TO hive;
        \end{minted}

        \item[Exit] \emph{psql} with the command: \mintinline{psql}{\q}
    \end{description}

    \subsection{Install \emph{Java Database Connectivity} Drivers}
    The \emph{Java Database Connectivity} (\emph{JDBC}) drivers allow \emph{Java} to interact with
    databases. As \emph{Hive} runs in \emph{Java}, it requires the \emph{PostgreSQL JDBC} drivers.
    Install the drivers drivers in \emph{Ubuntu} via the following command:
    \begin{minted}{bash}
        sudo apt install libpostgresql-jdbc-java
    \end{minted}


\end{document}