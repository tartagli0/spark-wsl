\documentclass{article}

\usepackage{minted}

\begin{document}

    \title{Creating a Stand-Alone Spark Environment in Windows Subsystem for Linux}
    \author{Abraham Vargas}
    \maketitle

    \newpage
    \tableofcontents
    \newpage

    \section{Prerequisites}

        \subsection{Java}
        \emph{Spark}, \emph{Hadoop}, and \emph{Hive} all require a \emph{Java Runtime Environment}(\emph{JRE}).
        \emph{OpenJDK} is a standard package in most \emph{Linux} distributions. Though \emph{OpenJDK 11}
        is the latest version, \emph{Spark} and its related components require \emph{OpenJDK 8}.
        
            \subsubsection{Install \emph{JRE}}
            To install \emph{OpenJDK 8}, run the following command in a terminal:
            \begin{minted}{bash}
            sudo apt install openjdk-8-jre-headless            
            \end{minted}

            \subsubsection{Set \emph{JAVA\_HOME}}
            \emph{Spark} and other components will need to know the path of the
            \emph{JRE}. This is accomplished by setting the \emph{JAVA\_HOME}
            system variable.

            \begin{description}

                \item[Open] the \emph{.bashrc} file with any editor (e.g., \emph{vim}, \emph{nano}).
                For example:
                \begin{minted}{bash}
                vim ~/.bashrc
                \end{minted}

                \item[Add] the following line to the \emph{.bashrc} file:
                \begin{minted}{bash}
                export $JAVA_HOME=/usr/lib/jvm/java-8-openjdk-amd64                 
                \end{minted}
                
                \item[Reload] shell environment configuration:
                \begin{minted}{bash}
                    source ~/.bashrc
                \end{minted}

                \item[Verify] that path was correctly set:
                \begin{minted}{bash}
                echo $JAVA_HOME
                \end{minted}
            \end{description}

\end{document}