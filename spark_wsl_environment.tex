\documentclass{article}

\usepackage{minted}
\usepackage[hyphens]{url}
\usepackage{microtype}
\usepackage{hyperref}
\hypersetup{
    colorlinks=true,
    linkcolor=blue,
    filecolor=magenta,      
    urlcolor=cyan
}

\begin{document}

\title{Creating a Stand-Alone Spark Environment in Windows Subsystem for Linux}
\author{Abraham Vargas}
\date{}
\maketitle

\begin{abstract}
Running a local \emph{Spark} cluster has multiple advantages. First, it facilitates unit testing.
This can decrease development time tenfold or more for larger applications. By running a
local \emph{Spark} cluster on a laptop or PC, smaller testing datasets can be created and multiple
\emph{Spark} configuration options can be experimented with without having to run tests on huge
datasets through industrial clusters. Second, having a local setup allows for learning
\emph{Spark}, its interfaces and syntax, and optimizations. One can learn \emph{Spark} without
having to purchase or sign up for online services (e.g., \emph{Amazon AWS}, \emph{Microsoft Azure},
\emph{Databricks}). Other advantages of running a local \emph{Spark} cluster include tweaking
options for optimization of production clusters and learning related technologies (e.g.,
\emph{Hadoop}, \emph{Hive}). Setting up a local \emph{Spark} environment is best performed in
\emph{Linux}-based systems, as most Big Data technologies are developed for use in such systems.
\emph{Windows Subsystem for Linux} (\emph{WSL}) provides a way to run a \emph{Linux} system
concurrently with \emph{Microsoft Windows}. This avoids having to install a full \emph{Linux}
distribution, learning the new system, and setting up a multi-boot PC. This guide will utilize
the \emph{Ubuntu 18.04 Linux} distribution within \emph{WSL} to teach the process of setting up a
local \emph{Spark} cluster.
\end{abstract}

\newpage
\tableofcontents
\newpage

\section{Prerequisites}
Before installing \emph{Spark} and its related components \emph{Hadoop} and \emph{Hive}, several
software, configuration, and other requirements must be met.

  \subsection{WSL 2}
  \emph{WSL} allows a \emph{Linux} distribution (distro) to run concurrently with \emph{Windows 10}.
  \emph{WSL 2} now includes a \emph{Linux} kernel and is several times faster than the original
  \emph{WSL}. This guide assumes that \emph{WSL 2} is installed and is running the
  \emph{Ubuntu 18.04} distro. To install \emph{WSL 2}, follow the
  \href{https://docs.microsoft.com/en-us/windows/wsl/install-win10}{Windows Subsystem
  for Linux Installation Guide for Windows 10}. Make sure to use \emph{Ubuntu 18.04} as the distro.

  \subsection{Java}
  \emph{Spark}, \emph{Hadoop}, and \emph{Hive} all require a \emph{Java Runtime Environment}
  (\emph{JRE}). \emph{OpenJDK} is the standard \emph{Java} platform for most \emph{Linux}
  distributions. Though \emph{OpenJDK 11} is the latest version, \emph{Spark} and its related
  components require \emph{OpenJDK 8}.

    \subsubsection{Install \emph{JRE}}
    To install \emph{OpenJDK 8}, run the following command in a terminal:
    \begin{minted}{bash}
    sudo apt install openjdk-8-jre-headless openjdk-8-jdk-headless  
    \end{minted}

    \subsubsection{Set \emph{JAVA\_HOME}}
    \label{subsec:javahome}
    \emph{Spark} and other components will need to know the path of the
    \emph{JRE}. This is accomplished by setting the \emph{JAVA\_HOME}
    system variable.

    \begin{enumerate}
    \item Open the \emph{.bashrc} file with any editor (e.g., \emph{vim}, \emph{nano}).
    For example:
    \begin{minted}{bash}
    vim ~/.bashrc
    \end{minted}
    
    \item Add the following line to the \emph{.bashrc} file:
    \label{itm:javahome}
    \begin{minted}{bash}
    export JAVA_HOME=/usr/lib/jvm/java-8-openjdk-amd64                 
    \end{minted}
    
    \item Reload shell environment configuration:
    \begin{minted}{bash}
    source ~/.bashrc
    \end{minted}
    
    \item Verify that path was correctly set:
    \begin{minted}{bash}
    echo $JAVA_HOME
    \end{minted}
    The output should match that in step \ref{itm:javahome}.
    \end{enumerate}

  \subsection{PostgreSQL}
  \label{subsec:postgres}
  \emph{Hive} requires a database for its \emph{metastore}. In terms of open-source options, \emph{Hive} can use
  \emph{MySQL}, \emph{PostgreSQL}, and \emph{Derby}. This guide will use \emph{PostgreSQL}.

    \subsubsection{Install \emph{PostgreSQL}}
    \emph{PostgreSQL 12} is the default version packaged for \emph{Ubuntu 18.04}. Install
    \emph{PostgreSQL 12} with the command:
    \begin{minted}{bash}
    sudo apt install postgresql-12
    \end{minted}

    \subsubsection{Start database server}
    \label{subsec:pgsstart}
    The \emph{PostgreSQL} will need to be started every time \emph{Hive} is started. Initialize the
    server using the command:
    \begin{minted}{bash}
    sudo service postgresql start
    \end{minted}

    \subsubsection{Configure \emph{Hive metastore}}
    The \emph{psql} application is used to interact with \emph{PostgreSQL} in a terminal.
    \begin{enumerate}
    \item Log into \emph{PostgreSQL} via \emph{psql} with default user \emph{postgres}:
    \begin{minted}{bash}
    sudo -u postgres psql 
    \end{minted}

    \item Add new user \emph{hive}
    (for simplicity, the new user will also be given \emph{hive} as the password):
    \begin{minted}{postgres}
    CREATE USER hive WITH PASSWORD 'hive';
    \end{minted}

    \item Create new database for \emph{Hive metastore}:
    \begin{minted}{postgres}
    CREATE DATABASE hive_metastore;
    \end{minted}

    \item Give ownership of \emph{hive\_metastore} database to user \emph{hive}:
    \begin{minted}{postgres}
    ALTER DATABASE hive_metastore OWNER TO hive;
    \end{minted}

    \item Exit \emph{psql} with the command: \mintinline{psql}{\q}
    \end{enumerate}

    \subsubsection{Install \emph{Java Database Connectivity} Drivers}
    The \emph{Java Database Connectivity} (\emph{JDBC}) drivers allow \emph{Java} to interact with
    databases. As \emph{Hive} runs in \emph{Java}, it requires the \emph{PostgreSQL JDBC} drivers.
    Install the drivers in \emph{Ubuntu} via the following command:
    \begin{minted}{bash}
    sudo apt install libpostgresql-jdbc-java
    \end{minted}

  \subsection{SSH}
  \emph{Secure Shell} (\emph{SSH}) encrypts communications over insecure networks. To accomplish
  encryption, \emph{SSH} utilizes public and private keys.

    \subsubsection{Generate key}
    Generate an \emph{ssh key} with the following command:
    \begin{minted}{bash}
    ssh-keygen -t rsa
    \end{minted}
    Leave the password blank when prompted, as \emph{Hadoop} will utilize a password-less
    \emph{SSH} login.

    \subsubsection{Allow \emph{localhost} login}
    \emph{Hadoop} requires the ability to log into the \emph{localhost} (i.e., your local PC) via
    password-less \emph{SSH}. To accomplish this, run the following command:
    \begin{minted}{bash}
    cat ~/.ssh/id_rsa.pub >> ~/.ssh/authorized_keys
    \end{minted}

    \subsubsection{Restart \emph{SSH} service}
    \label{subsec:sshrestart}
    Restart \emph{SSH} and apply new settings:
    \begin{minted}{bash}
    sudo service ssh restart
    \end{minted}
    \textbf{Note}: This command must be re-executed any time the following error message is
    encountered:
    \begin{minted}{bash}
    ssh: connect to host localhost port 22: Connection refused
    \end{minted}

    \subsubsection{\emph{Could not load host key} error}
    If attempting to restart \emph{SSH} service returns \emph{Could not load host key} errors, run
    the following command:
    \begin{minted}{bash}
    ssh-keygen -A
    \end{minted}
    Rerunning the command in \ref{subsec:sshrestart} should now complete without errors.

    \subsubsection{Verify password-less \emph{SSH} login}
    Ensure that \emph{localhost} can be accessed via \emph{SSH} without a password:
    \begin{minted}{bash}
    ssh localhost
    \end{minted}

\newpage
\section{Hadoop}
\emph{Apache Hadoop} is a framework that is used to store and process \emph{big data} in
distributed systems. In this setup guide, \emph{Hadoop} will be used as a data storage system on a
single PC.

  \subsection{Download and extract \emph{Hadoop 3.3.0}}
  This guide will use the latest stable version, \emph{Hadoop 3.3.0}.
  
    \subsubsection{Download \emph{Hadoop}}
    Run the \emph{wget} command in a terminal to download:
    \begin{minted}{bash}
    wget https://apache.osuosl.org/hadoop/common/hadoop-3.3.0/hadoop-3.3.0.tar.gz
    \end{minted}

    \subsubsection{Extract files}
    Run the following command to extract the \emph{tar.gz} file:
    \begin{minted}{bash}
    tar -xvzf hadoop-3.3.0.tar.gz
    \end{minted}
    A new directory, \path{hadoop-3.3.0}, should now appear.

    \subsubsection{Copy folder to \emph{/opt}}
    In \emph{Ubuntu}, \path{/opt} is the directory used to store add-on applications.
    \emph{Hadoop}, \emph{Hive}, and \emph{Spark} will all be stored in this directory. Run the
    following command to move the extracted \emph{Hive} folder:
    \begin{minted}{bash}
    sudo mv hadoop-3.3.0 /opt
    \end{minted}
    
  \subsection{Set \emph{HADOOP\_HOME}}
  Setting the \emph{HADOOP\_HOME} variable tells the system where \emph{HADOOP} is located.

    \subsubsection{Edit \emph{.bashrc}}
    \label{subsec:hadoophome}
    Open the \emph{$\sim$/.bashrc} file and add the following line:
    \begin{minted}{bash}
    export HADOOP_HOME=/opt/hadoop-3.3.0
    \end{minted}

    \subsubsection{Reload configuration}
    \label{subsec:reloadbash}
    Load the new \emph{bash} configuration with the command:
    \begin{minted}{bash}
    source ~/.bashrc
    \end{minted}

    \subsubsection{Verify path}
    Verify that \emph{HADOOP\_HOME} was correctly set:
    \begin{minted}{bash}
    echo $HADOOP_HOME
    \end{minted}
    The output should match the path in \ref{subsec:hadoophome}.
    
  \subsection{Add \emph{Hadoop} to system path}
  In order to run \emph{Hadoop} commands without having to specify a full path, e.g.,
  \begin{minted}{bash}
  /opt/hadoop-3.3.0/bin/hadoop fs -mkdir /tmp
  \end{minted}
  vs.
  \begin{minted}{bash}
  hadoop fs -mkdir /tmp
  \end{minted}
  add the following line to $\sim$/.bashrc:
  \begin{minted}{bash}
  PATH=$PATH:$HADOOP_HOME/bin
  \end{minted}
  Make sure to reload the \emph{Bash} configuration as in section \ref{subsec:reloadbash}.

  \subsection{Edit configuration files}
  In this section, \emph{Hadoop} will be configured to run in single-node mode
  (i.e., on a single PC).
    \subsubsection{Edit \emph{hadoop-env.sh}}
    Add the \emph{JAVA\_HOME} variable to the \emph{Hadoop} environment, as in section
    \ref{subsec:javahome}.
    \begin{enumerate}
    \item Open the file \path{/opt/hadoop-3.3.0/etc/hadoop/hadoop-env.sh} with any editor
    
    \item Un-comment line \emph{54} and add the correct path:
    \begin{minted}{bash}
    # The java implementation to use. By default, this environment
    # variable is REQUIRED on ALL platforms except OS X!
    export JAVA_HOME=/usr/lib/jvm/java-8-openjdk-amd64
    \end{minted}
    \end{enumerate}

    \subsubsection{Edit \emph{core-site.xml}}
    Edit the file \path{/opt/hadoop-3.3.0/etc/hadoop/core-site.xml} to look like the example below:
    \begin{minted}{xml}
    <configuration>
      <property>
        <name>fs.defaultFS</name>
        <value>hdfs://localhost:9000</value>
      </property>
      <property>
        <name>hadoop.proxyuser.abe.hosts</name>
        <value>*</value>
      </property>
      <property>
        <name>hadoop.proxyuser.abe.groups</name>
        <value>*</value>
      </property>
    </configuration>
    \end{minted}
    Replace \emph{abe} above with your own \emph{WSL Linux} username.

    \subsubsection{Edit \emph{hdfs-site.xml}}
    Edit the file \path{/opt/hadoop-3.3.0/etc/hadoop/hdfs-site.xml} to look like the example below:
    \begin{minted}{xml}
    <configuration>
      <property>
        <name>dfs.replication</name>
        <value>1</value>
      </property>
    </configuration>
    \end{minted}
        
    \subsubsection{Edit \emph{mapred-site.xml}}
    Edit the file \path{/opt/hadoop-3.3.0/etc/hadoop/mapred-site.xml} to look like the example
    below:
    \begin{minted}{xml}
    <configuration>
      <property>
        <name>mapreduce.framework.name</name>
        <value>yarn</value>
      </property>
      <property>
        <name>mapreduce.application.classpath</name>
        <value>
          $HADOOP_MAPRED_HOME/share/hadoop/mapreduce/*:
          $HADOOP_MAPRED_HOME/share/hadoop/mapreduce/lib/*
        </value>
      </property>
    </configuration>
    \end{minted}
        
    \subsubsection{Edit \emph{yarn-site.xml}}
    Edit the file \path{/opt/hadoop-3.3.0/etc/hadoop/yarn-site.xml} to look like the example below:
    \begin{minted}{xml}
    <configuration>
      <property>
        <name>yarn.nodemanager.aux-services</name>
        <value>mapreduce_shuffle</value>
      </property>
      <property>
        <name>yarn.nodemanager.env-whitelist</name>
        <value>
          JAVA_HOME,HADOOP_COMMON_HOME,HADOOP_HDFS_HOME,
          HADOOP_CONF_DIR,CLASSPATH_PREPEND_DISTCACHE,
          HADOOP_YARN_HOME,HADOOP_MAPRED_HOME
        </value>
      </property>
    </configuration>
    \end{minted}

  \subsection{Format \emph{HDFS}}
  Format the \emph{Hadoop Distributed File System} (\emph{HDFS}) with the command:
  \begin{minted}{bash}
  hdfs namenode -format
  \end{minted}
  The \emph{hdfs} command should warn that the log file does not exist. This will be followed
  by several lines of \emph{INFO} messages.

  \subsection{Start \emph{NameNode} and \emph{DataNode} daemons}
  The \emph{NameNode} and \emph{DataNode} daemons generate system stats that can be viewed
  through a web interface. To start both daemons, run the command:
  \begin{minted}{bash}
  $HADOOP_HOME/sbin/start-dfs.sh
  \end{minted}
  The above command might return a \emph{connection refused} error. If this occurs, restart
  the \emph{SSH} service (see section \ref{subsec:sshrestart}). The web interface should now be
  accessible via \url{http://localhost:9870/}.

  \subsection{Start \emph{YARN} daemon}
  \emph{YARN} acts as a resource manager in a \emph{Hadoop} system. To start the \emph{YARN}
  daemon, run the command:
  \begin{minted}{bash}
  $HADOOP_HOME/sbin/start-yarn.sh
  \end{minted}
  The resource manager web interface should now be accessible via
  \url{http://localhost:8088/}.

\newpage
\section{Hive}
\emph{Apache Hive} is a data warehousing platform that utilizes \emph{SQL} to interact with
data residing in distributed systems. \emph{Hive} requires a \emph{metastore} service, which in this
case is provided by \emph{PostgreSQL} (section \ref{subsec:postgres}).

  \subsection{Download and extract \emph{Hive}}
  This guide will use the current latest stable version, \emph{Apache Hive 3.1.2}.
  \begin{enumerate}
  \item Download \emph{Hive} via the \emph{wget} terminal command:
  \begin{minted}[autogobble]{bash}
  wget http://apache.mirrors.pair.com/hive/hive-3.1.2/apache-hive-3.1.2-bin.tar.gz
  \end{minted}

  \item Unpack \emph{Hive} file using the \emph{tar} command:
  \begin{minted}{bash}
  tar -xvzf apache-hive-3.1.2-bin.tar.gz
  \end{minted}

  \item Move new folder to /opt directory:
  \begin{minted}{bash}
  sudo mv apache-hive-3.1.2-bin /opt
  \end{minted}
  \end{enumerate}

  \subsection{Set \emph{HIVE\_HOME}}
  The \emph{HIVE\_HOME} variable must be set in order to let the system know \emph{Hive's}
  location.
  \begin{enumerate}
  \item Open $\sim$/.bashrc and add the following line:
  \label{itm:hivehome}
  \begin{minted}{bash}
  export HIVE_HOME=/opt/apache-hive-3.1.2-bin
  \end{minted}

  \item Also add the following line to include \emph{Hive} in the system path:
  \begin{minted}{bash}
  PATH=$PATH:$HIVE_HOME/bin
  \end{minted}

  \item Reload \emph{Bash} terminal configuration:
  \begin{minted}{bash}
  source ~/.bashrc
  \end{minted}
  
  \item Ensure \emph{HIVE\_HOME} path has been correctly set:
  \begin{minted}{bash}
  echo $HIVE_HOME
  \end{minted}
  Output should display the path configured in step \ref{itm:hivehome} above.
  \end{enumerate}

  \subsection{Initialize all \emph{Hadoop} services}
  \label{subsec:hadoopall}
  As \emph{Hive} manages data in a \emph{Hadoop} system, the latter must be running.
  Run the following command in a terminal to start all \emph{Hadoop} services:
  \begin{minted}{bash}
  $HADOOP_HOME/sbin/start-all.sh
  \end{minted}
  The above command should warn that the configuration is not recommended for production.
  If \emph{connection refused} errors are returned, restart \emph{SSH} service as
  in section \ref{subsec:sshrestart}.

  \subsection{Configure \emph{HDFS}}
  The \emph{HDFS} directories that \emph{Hive} will use must first be created and configured.
  Run the following commands to create the required \emph{Hive} folders and assign appropriate
  permissions:
  \begin{minted}{bash}
  hadoop fs -mkdir /tmp 
  hadoop fs -mkdir -p /user/hive/warehouse 
  hadoop fs -chmod g+w /tmp 
  hadoop fs -chmod g+w /user/hive/warehouse
  \end{minted}

  \subsection{Edit \emph{hive-site.xml}}
  The \emph{hive-site.xml} file contains required configuration settings for the
  \emph{Hive metastore}. In this setup, the \emph{metastore} resides within a \emph{PostgreSQL}
  database (see section \ref{subsec:postgres}).

    \subsubsection{Create \emph{hive-site.xml}}
    Copy the file \emph{hive-default.xml.template} and rename it as \emph{hive-site.xml}:
    \begin{minted}{bash}
    cd /opt/apache-hive-3.1.2-bin/conf
    cp hive-default.xml.template hive-site.xml
    \end{minted}
        
    \subsubsection{Edit configuration settings}
    Edit the following values in the \emph{hive-site.xml} file at the lines shown below. Note
    that \emph{\textless value/\textgreater} must be replace with
    \emph{\textless /value\textgreater} in \emph{line 462}.
    \begin{minted}[numbers=left, firstnumber=461]{xml}
    <name>hive.metastore.uris</name>
    <value>thrift://127.0.0.1:9083</value>
    \end{minted}
    \centerline{\vdots}
    \begin{minted}[numbers=left, firstnumber=568]{xml}
    <name>javax.jdo.option.ConnectionPassword</name>
    <value>hive</value>
    \end{minted}
    \centerline{\vdots}
    \begin{minted}[numbers=left, firstnumber=583]{xml}
    <name>javax.jdo.option.ConnectionURL</name>
    <value>jdbc:postgresql://127.0.0.1/hive_metastore</value>
    \end{minted}
    \centerline{\vdots}
    \begin{minted}[numbers=left, firstnumber=1101]{xml}
    <name>javax.jdo.option.ConnectionDriverName</name>
    <value>org.postgresql.Driver</value>
    \end{minted}
    \centerline{\vdots}
    \begin{minted}[numbers=left, firstnumber=1126]{xml}
    <name>javax.jdo.option.ConnectionUserName</name>
    <value>hive</value>
    \end{minted}

    \subsubsection{Remove illegal characters}
    Line 3215 in \emph{hive-site.xml} contains illegal characters that will cause a
    \emph{Java RuntimeException}. Remove the characters highlighted below from line 3215:
    \begin{minted}[autogobble, escapeinside=||]{xml}
      <property>
        <name>hive.txn.xlock.iow</name>
        <value>true</value>
        <description>|\dots|locks for|\colorbox{yellow}{&#8;}||\dots|OVERWRITE.</description>
      </property>
    \end{minted}
        
  \subsection{Create \emph{PostgreSQL} database structure}
  The \emph{Hive Schema Tool} will create a database structure for the \emph{Hive metastore}.
  Before this can happen, a few configurations must be adjusted.
        
    \subsubsection{Update \emph{guava} version}
    The versions of \emph{guava} between \emph{Hadoop 3.3.0} and \emph{Hive 3.1.2} are not
    compatible. Having different versions will result in a \emph{Java NoSuchMethodError} in step
    \ref{subsec:schematool}.
    \begin{enumerate}
    \item Delete \emph{guava 19.0} from \emph{Hive}:
    \begin{minted}{bash}
    cd /opt/apache-hive-3.1.2-bin
    rm lib/guava-19.0.jar
    \end{minted}
    
    \item Replace with \emph{guava 27.0} included with \emph{Hadoop}:
    \begin{minted}[autogobble]{bash}
    cp /opt/hadoop-3.3.0/share/hadoop/hdfs/lib/guava-27.0-jre.jar lib/
    \end{minted}
    \end{enumerate}
        
    \subsubsection{Run \emph{Schema Tool}}
    \label{subsec:schematool}
    The command below will create the database structure for the \emph{Hive metastore}. Once
    successfully completed, the output should conclude with the message
    \emph{schemaTool completed}.
    \begin{minted}{bash}
    schematool -dbType postgres -initSchema
    \end{minted}
        
  \subsection{Start \emph{Hive} services}
  \label{subsec:hiveservices}
  In order to use \emph{Hive}, its \emph{metastore} and server must be started.
    \subsubsection{Prevent \emph{URISyntaxException}}
    To prevent a \emph{Java URISyntaxException} from being returned when attempting to run
    \emph{Hive} services, add the following lines directly after \emph{line 21} in
    \emph{hive-site.xml}:
    \begin{minted}{xml}
    <property>
      <name>system:java.io.tmpdir</name>
      <value>/tmp/hive/java</value>
    </property>
    <property>
      <name>system:user.name</name>
      <value>${user.name}</value>
    </property>
    \end{minted}

    \subsubsection{Start \emph{Hive metastore} service}
    Run the following command in a terminal to start \emph{Hive metastore} services:
    \begin{minted}{bash}
    hive --service metastore
    \end{minted}

    \subsubsection{Start \emph{Hive} server}
    Open another terminal and run the following command to Initialize the \emph{Hive} server:
    \begin{minted}{bash}
    hive --service hiveserver2
    \end{minted}

    \subsubsection{Access web interface}
    \emph{Hive} also provides a web interface with information on running sessions, queries, etc.
    After about \emph{one minute} of running both \emph{metastore} and \emph{Hive} server, access
    the web interface via \url{http://localhost:10002/}

\newpage
\section{Spark}
With \emph{Hadoop} and \emph{Hive} configured, we can finally set up and use \emph{Spark}.

  \subsection{Download and extract \emph{Spark}}
  This guide will use \emph{Spark 3.0.1}. As \emph{Hadoop 3.3.0} is
  already installed, we'll download \emph{Spark Pre-built for Apache Hadoop 3.2 and later}.
  \begin{enumerate}
  \item Use the \emph{wget} command to download \emph{Spark 3.0.1}:
  \begin{minted}[autogobble]{bash}
  wget https://archive.apache.org/dist/spark/spark-3.0.1/spark-3.0.1-bin-hadoop3.2.tgz
  \end{minted}

  \item Extract the downloaded file:
  \begin{minted}{bash}
  tar -xvzf spark-3.0.1-bin-hadoop3.2.tgz
  \end{minted}

  \item Move \emph{Spark} folder into \emph{/opt} directory:
  \begin{minted}{bash}
  sudo mv spark-3.0.1-bin-hadoop3.2 /opt
  \end{minted}
  \end{enumerate}

  \subsection{Set \emph{SPARK\_HOME} variable}
  Edit $\sim$/.bashrc and add the following line:
  \begin{minted}{bash}
  export SPARK_HOME=/opt/spark-3.0.1-bin-hadoop3.2
  \end{minted}
  
  \subsection{Add \emph{Spark} to system path}
  Adding \emph{Spark} to the system path will allow you to run \emph{Spark} commands without
  using full paths. Add the following line to $\sim$/.bashrc:
  \begin{minted}{bash}
  PATH=$PATH:$SPARK_HOME/bin
  \end{minted}
  
  \subsection{Set local IP}
  \emph{Ubuntu} adds a \emph{hostname} entry to the \emph{/etc/hosts} file, which can cause
  conflicts with \emph{localhost} when \emph{Spark} attempts to resolve IP addresses. To avoid
  IP-related warnings and errors, add the following line to $\sim$/.bashrc:
  \begin{minted}{bash}
  export SPARK_LOCAL_IP=localhost
  \end{minted}

  \subsection{Reload configuration}
  Reload the new \emph{Bash} shell configuration:
  \begin{minted}{bash}
  source ~/.bashrc
  \end{minted}

  \subsection{Enable \emph{Hive} support}
  In order to use \emph{Hive} with \emph{Spark}, we need to copy configuration files from both
  \emph{Hive} and \emph{Hadoop}. Run the following commands to copy the necessary files to
  \emph{Spark}:
  \begin{minted}{bash}
  cp $HADOOP_HOME/etc/hadoop/core-site.xml $SPARK_HOME/conf/
  cp $HADOOP_HOME/etc/hadoop/hdfs-site.xml $SPARK_HOME/conf/
  cp $HIVE_HOME/conf/hive-site.xml $SPARK_HOME/conf/
  \end{minted}

  \subsection{Enable \emph{Spark} logs web interface}
  \emph{Spark} provides a web interface from which event logs can be viewed.
    \subsubsection{Edit configuration file}
    In order to activate the logging capability of \emph{Spark}, the \emph{spark-defaults.conf}
    file must be created and edited.
    \begin{enumerate}
    \item Copy default \emph{Spark} configuration file:
    \begin{minted}{bash}
    cd $SPARK_HOME/conf
    cp spark-defaults.conf.template spark-defaults.conf
    \end{minted}

    \item Add the following lines to \emph{spark-defaults.conf}:
    \begin{minted}{text}
    spark.eventLog.enabled           true
    spark.eventLog.dir               hdfs://localhost:9000/spark-logs
    spark.history.fs.logDirectory    hdfs://localhost:9000/spark-logs
    \end{minted}
    \end{enumerate}

    \subsubsection{Create log directory in \emph{HDFS}}
    The logs generated by \emph{Spark} will be saved to a \emph{HDFS} folder. Create the folder
    and assign appropriate permissions using the commands below:
    \begin{minted}{bash}
    hadoop fs -mkdir /spark-logs
    hadoop fs -chmod g+w /spark-logs
    \end{minted}

    \subsubsection{Start \emph{Spark} history server}
    \label{subsec:sparklog}
    Run the following command to start \emph{Spark} history log server:
    \begin{minted}{bash}
    $SPARK_HOME/sbin/start-history-server.sh
    \end{minted}

    \subsubsection{Access web interface}
    The web interface can now be accessed via \url{http://localhost:18080/}. The page will
    display the message \emph{No completed applications found!} the first time it's accessed.
    This message can be safely ignored and will not appear after Spark jobs are run.

  \subsection{\emph{Spark} Shells}
  \emph{Spark} provides multiple shells that utilize different programming and scripting languages.

    \subsubsection{Scala}
    The default shell uses \emph{Scala} as a \emph{Spark} interface. \emph{Scala} is included
    with \emph{Spark} and only requires a \emph{JRE}. Entering the command below will start
    the default \emph{Scala} shell:
    \begin{minted}{bash}
    spark-shell
    \end{minted}

    \subsubsection{Spark SQL}
    \emph{Spark} includes \emph{Spark SQL}, which utilizes \emph{HiveSQL} syntax. To start
    a \emph{Spark SQL} shell, run the command:
    \begin{minted}{bash}
    spark-sql
    \end{minted}

    \subsubsection{Python}
    \emph{PySpark} is \emph{Spark's} interface for the \emph{Python} scripting language.
    Before using \emph{PySpark}, \emph{Python} must first be installed and configured.
    These tasks are beyond the scope of this documentation. The command below will initialize
    a \emph{Python} shell:
    \begin{minted}{bash}
    pyspark
    \end{minted}

    \subsubsection{R}
    \emph{Spark} also has an \emph{R} interface, \emph{SparkR}. Before utilizing the \emph{SparkR}
    shell, \emph{R} must first be installed and configured, which is beyond the scope of this
    documentation. Run the following command to start a \emph{SparkR} shell:
    \begin{minted}{bash}
    sparkR
    \end{minted}

\newpage
\section{Putting It All Together}
Now that \emph{Spark} and all other necessary components are installed and configured, they must be
activated every time when starting a new \emph{Windows} session. Activating all components can be
accomplished manually each time, or by the use of a script for automation.

  \subsection{Option 1: Manual activation}
  To manually start all components required for using a \emph{Spark} shell, the following steps
  must be performed in order:
  \begin{enumerate}
  \item Start \emph{PostgreSQL} server (section \ref{subsec:pgsstart})
  \begin{minted}{bash}
  sudo service postgresql start
  \end{minted}

  \item Restart \emph{SSH} service (section \ref{subsec:sshrestart})
  \begin{minted}{bash}
  sudo service ssh restart
  \end{minted}

  \item Start all \emph{Hadoop} services (section \ref{subsec:hadoopall})
  \begin{minted}{bash}
  $HADOOP_HOME/sbin/start-all.sh
  \end{minted}

  \item Start \emph{Hive} \emph{metastore} and server (section \ref{subsec:hiveservices}).
  The commands below must be run in \textbf{new} and \textbf{separate} terminals:
  \begin{minted}{bash}
  hive --service metastore
  hive --service hiveserver2
  \end{minted}

  \item Start \emph{Spark} history server (section \ref{subsec:sparklog}):
  \begin{minted}{bash}
  $SPARK_HOME/sbin/start-history-server.sh
  \end{minted}
  \end{enumerate}

  \subsection{Option 2: Script automation}
  As an alternative to performing the steps above every time before being able to use \emph{Spark},
  you can create a simple \emph{Bash} shell script.
  \begin{enumerate}
  \item Open an editor and enter the following lines:
  \begin{minted}{bash}
  #! /bin/bash

  sudo service postgresql start
  sudo service ssh restart
  $HADOOP_HOME/sbin/start-all.sh
  nohup hive --service metastore > .hivemetastore &
  nohup hive --service hiveserver2 > .hiveserver &
  $SPARK_HOME/sbin/start-history-server.sh
  \end{minted}
  Notice that the script above uses \href{https://en.wikipedia.org/wiki/Nohup}{nohup}. This is to
  avoid having to open multiple terminals for the \emph{Hive} services to run in.
  
  \item Save the script as \emph{start-all-spark} (or any other name you'd like to give it)

  \item Make the file executable with the command:
  \begin{minted}{bash}
  chmod +x start-all-spark
  \end{minted}

  \item Run the script to start all services required for using \emph{Spark}:
  \begin{minted}{bash}
  ./start-all-spark
  \end{minted}
\end{enumerate}

\newpage
\begin{thebibliography}{9}
  \bibitem{postgres_metastore}
  Hewlett Packard Enterprise Development LP,
  \textit{Configuring a Remote PostgreSQL Database for the Hive Metastore},
  2020,
  \url{https://docs.datafabric.hpe.com/61/Hive/Config-RemotePostgreSQLForHiveMetastore.html}
  
  \bibitem{hadoop_install}
  The Apache Software Foundation,
  \textit{Hadoop: Setting up a Single Node Cluster},
  2020,
  \url{
    https://hadoop.apache.org/docs/r3.3.0/hadoop-project-dist/hadoop-common/
    SingleCluster.html}

  \bibitem{ubuntu_install}
  Debian Installer Team,
  \textit{Ubuntu Installation Guide},
  2020,
  \url{https://help.ubuntu.com/lts/installation-guide/amd64/index.html}

  \bibitem{guava_error}
  Pradeep Kumar, Javi Roman,
  \textit{Hive Issue 22915},
  2020,
  \url{https://issues.apache.org/jira/browse/HIVE-22915}

  \bibitem{hive}
  Raymond Tang,
  \textit{Apache Hive 3.1.1 Installation on Windows 10 using Windows Subsystem for Linux},
  2018,
  \url{
    https://kontext.tech/column/hadoop/309/apache-hive-311-installation-on-windows-10-using-
    windows-subsystem-for-linux}

  \bibitem{hive_error_fix}
  Vu Duc Tiep,
  \textit{[Hive Installation] java.net.URISyntaxException: Relative path in absolute URI},
  2017,
  \url{http://driftingengineer.blogspot.com/2017/09/hive-installation-javaneturisyntaxexcep.html}

  \bibitem{spark}
  Raymond Tang,
  \textit{Apache Spark 2.4.3 Installation on Windows 10 using Windows Subsystem for Linux},
  2018,
  \url{
    https://kontext.tech/column/spark/311/apache-spark-243-installation-on-windows-10-using-
    windows-subsystem-for-linux}

  \bibitem{spark-env-var}
  The Apache Software Foundation,
  \textit{Spark Configuration: Environment Variables},
  2021,
  \url{https://spark.apache.org/docs/latest/configuration.html#environment-variables}

  \bibitem{connect-refused}
  The Apache Software Foundation,
  \textit{HADOOP2: Connection Refused},
  2019,
  \url{https://cwiki.apache.org/confluence/display/HADOOP2/ConnectionRefused}
\end{thebibliography}

\end{document}